\chapter{Opcode Map}
\label{appendix:opcode-map}

\section{Complete Opcode Reference}

This appendix provides a complete map of all 64 possible opcodes in the SIRCIS instruction set.

\begin{table}[H]
\centering
\footnotesize
\begin{tabular}{|c|l|l|l|}
\hline
\textbf{Hex} & \textbf{Binary} & \textbf{Mnemonic} & \textbf{Description} \\
\hline
\multicolumn{4}{|c|}{\textbf{ALU Immediate (0x00--0x0F)}} \\
\hline
00 & 000000 & ADDI & Add Immediate \\
01 & 000001 & ADCI & Add with Carry Immediate \\
02 & 000010 & SUBI & Subtract Immediate \\
03 & 000011 & SBCI & Subtract with Carry Immediate \\
04 & 000100 & ANDI & AND Immediate \\
05 & 000101 & ORRI & OR Immediate \\
06 & 000110 & XORI & XOR Immediate \\
07 & 000111 & LOAD & Load Immediate \\
08 & 001000 & -- & Undocumented \\
09 & 001001 & -- & Undocumented \\
0A & 001010 & CMPI & Compare Immediate \\
0B & 001011 & -- & Undocumented \\
0C & 001100 & TSAI & Test AND Immediate \\
0D & 001101 & -- & Undocumented \\
0E & 001110 & TSXI & Test XOR Immediate \\
0F & 001111 & COPI & Coprocessor Immediate \\
\hline
\multicolumn{4}{|c|}{\textbf{Memory and Control (0x10--0x1F)}} \\
\hline
10 & 010000 & STOR & Store Indirect Immediate \\
11 & 010001 & STOR & Store Indirect Register \\
12 & 010010 & STOR & Store Pre-Decrement Immediate \\
13 & 010011 & STOR & Store Pre-Decrement Register \\
14 & 010100 & LOAD & Load Indirect Immediate \\
15 & 010101 & LOAD & Load Indirect Register \\
16 & 010110 & LOAD & Load Post-Increment Immediate \\
17 & 010111 & LOAD & Load Post-Increment Register \\
18 & 011000 & LDEA & Load Effective Address Immediate \\
19 & 011001 & LDEA & Load Effective Address Register \\
1A & 011010 & BRAN & Branch Immediate \\
1B & 011011 & BRAN & Branch Register \\
1C & 011100 & LJSR & Long Jump to Subroutine Immediate \\
1D & 011101 & LJSR & Long Jump to Subroutine Register \\
1E & 011110 & BRSR & Branch to Subroutine Immediate \\
1F & 011111 & BRSR & Branch to Subroutine Register \\
\hline
\multicolumn{4}{|c|}{\textbf{ALU Short Immediate (0x20--0x2F)}} \\
\hline
20 & 100000 & ADDI & Add Short Immediate \\
21 & 100001 & ADCI & Add with Carry Short Immediate \\
22 & 100010 & SUBI & Subtract Short Immediate \\
23 & 100011 & SBCI & Subtract with Carry Short Immediate \\
24 & 100100 & ANDI & AND Short Immediate \\
25 & 100101 & ORRI & OR Short Immediate \\
26 & 100110 & XORI & XOR Short Immediate \\
27 & 100111 & LOAD & Load Short Immediate \\
28 & 101000 & -- & Undocumented \\
29 & 101001 & -- & Undocumented \\
2A & 101010 & CMPI & Compare Short Immediate \\
2B & 101011 & -- & Undocumented \\
2C & 101100 & TSAI & Test AND Short Immediate \\
2D & 101101 & -- & Undocumented \\
2E & 101110 & TSXI & Test XOR Short Immediate \\
2F & 101111 & COPI & Coprocessor Short Immediate \\
\hline
\multicolumn{4}{|c|}{\textbf{ALU Register (0x30--0x3F)}} \\
\hline
30 & 110000 & ADDR & Add Register \\
31 & 110001 & ADCR & Add with Carry Register \\
32 & 110010 & SUBR & Subtract Register \\
33 & 110011 & SBCR & Subtract with Carry Register \\
34 & 110100 & ANDR & AND Register \\
35 & 110101 & ORRR & OR Register \\
36 & 110110 & XORR & XOR Register \\
37 & 110111 & LOAD & Load Register \\
38 & 111000 & -- & Undocumented \\
39 & 111001 & -- & Undocumented \\
3A & 111010 & CMPR & Compare Register \\
3B & 111011 & -- & Undocumented \\
3C & 111100 & TSAR & Test AND Register \\
3D & 111101 & -- & Undocumented \\
3E & 111110 & TSXR & Test XOR Register \\
3F & 111111 & COPR & Coprocessor Register \\
\hline
\end{tabular}
\caption{Complete SIRCIS Opcode Map}
\end{table}

\section{Opcode Patterns}

\subsection{Format Identification}

\begin{itemize}
    \item Bits 5--4 = 00: Immediate format (0x00--0x0F)
    \item Bits 5--4 = 01: Memory/Control operations (0x10--0x1F)
    \item Bits 5--4 = 10: Short Immediate with Shift (0x20--0x2F)
    \item Bits 5--4 = 11: Register format (0x30--0x3F)
\end{itemize}

\subsection{Save vs. Test}

\begin{itemize}
    \item Bit 3 = 0: Save result (0x00--0x07, 0x20--0x27, 0x30--0x37)
    \item Bit 3 = 1: Test only, discard result (0x08--0x0F, 0x28--0x2F, 0x38--0x3F)
\end{itemize}

\subsection{Memory Operations Decoding}

For opcodes 0x10--0x1F, bits encode memory operation type:

\begin{itemize}
    \item Bit 4 = 1: Always set for memory/control section
    \item Bits 3--2: Operation type
    \begin{itemize}
        \item 00 = Store
        \item 01 = Load
        \item 10 = Load Effective Address
        \item 11 = Load EA with Link (jump to subroutine)
    \end{itemize}
    \item Bit 1: Determines if both registers in address pair are updated
    \item Bit 0: Immediate (0) or Register (1) offset
\end{itemize}

\section{Undocumented Instructions}

The following opcodes are undocumented and should not be used in production code:

\begin{itemize}
    \item 0x08, 0x09, 0x0B, 0x0D (Immediate format, test variants)
    \item 0x28, 0x29, 0x2B, 0x2D (Short Immediate format, test variants)
    \item 0x38, 0x39, 0x3B, 0x3D (Register format, test variants)
\end{itemize}

These opcodes may execute in hardware but their behavior is implementation-defined and may change between CPU revisions. See Appendix~\ref{appendix:undocumented} for details.

\section{Quick Lookup Table}

\begin{table}[H]
\centering
\small
\begin{tabular}{llll}
\toprule
\textbf{Operation} & \textbf{Immediate} & \textbf{Short+Shift} & \textbf{Register} \\
\midrule
ADD & 0x00 & 0x20 & 0x30 \\
ADC & 0x01 & 0x21 & 0x31 \\
SUB & 0x02 & 0x22 & 0x32 \\
SBC & 0x03 & 0x23 & 0x33 \\
AND & 0x04 & 0x24 & 0x34 \\
OR & 0x05 & 0x25 & 0x35 \\
XOR & 0x06 & 0x26 & 0x36 \\
LOAD & 0x07 & 0x27 & 0x37 \\
CMP & 0x0A & 0x2A & 0x3A \\
TSA & 0x0C & 0x2C & 0x3C \\
TSX & 0x0E & 0x2E & 0x3E \\
COP & 0x0F & 0x2F & 0x3F \\
\bottomrule
\end{tabular}
\caption{Instruction Variant Quick Reference}
\end{table}
