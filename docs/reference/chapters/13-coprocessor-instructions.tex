\chapter{Coprocessor Instructions}
\label{ch:coprocessor}

\section{Overview}

The SIRC-1 CPU includes a coprocessor interface that allows delegation of certain operations to specialized
coprocessors. The primary coprocessor is the Exception Unit (Coprocessor 1), which handles exceptions, interrupts, and
privileged operations.

\begin{instructionbox}{COP -- Coprocessor Call}
	\textbf{Opcodes:} 0x0F (Imm), 0x2F (Short Imm), 0x3F (Reg)

	\textbf{Syntax:}
	\begin{lstlisting}
COPI #opcode                  ; Call coprocessor with 16-bit opcode
COPR rS                       ; Call coprocessor with register value
\end{lstlisting}

	\textbf{Description:}

	Transfers control to a coprocessor. The coprocessor ID and operation code are encoded in the immediate value or
	register. Coprocessor operations may be privileged.

	\textbf{Format:}
	\begin{itemize}
		\item Bits 15--12: Coprocessor ID
		\item Bits 11--0: Operation code
	\end{itemize}

	\textbf{Examples:}
	\begin{lstlisting}
; Wait for interrupt (Exception Unit)
COPI #0x1900                  ; Coprocessor 1, opcode 0x900
; Assembled as WAIT meta-instruction

; Return from exception
COPI #0x1A00                  ; Coprocessor 1, opcode 0xA00
; Assembled as RETE meta-instruction

; User exception (system call)
COPI #0x1180                  ; Coprocessor 1, opcode 0x180 (vector 0x80)
; Assembled as EXCP #0x80 meta-instruction
\end{lstlisting}

\end{instructionbox}

\section{Exception Unit (Coprocessor 1)}

The Exception Unit handles:
\begin{itemize}
	\item Hardware exceptions (faults, traps)
	\item Software exceptions
	\item Interrupts (multiple priority levels)
	\item Privileged operations
\end{itemize}

\subsection{Common Operations}

\begin{table}[H]
	\centering
	\begin{tabular}{llp{8cm}}
		\toprule
		\textbf{Opcode} & \textbf{Mnemonic} & \textbf{Operation}                        \\
		\midrule
		0x1100-0x11FF   & EXCP              & User exception (trap) at vector 0x60-0xFF \\
		0x1900          & WAIT              & Wait for interrupt                        \\
		0x1A00          & RETE              & Return from exception                     \\
		0x1B00          & RSET              & System reset                              \\
		0x1C00-0x1C7F   & ETFR              & Transfer from exception link register     \\
		0x1D00-0x1D7F   & ETTR              & Transfer to exception link register       \\
		\bottomrule
	\end{tabular}
	\caption{Exception Unit Operations}
\end{table}

See Chapter~\ref{ch:exceptions} for detailed documentation of exception handling and all exception coprocessor
instructions.
