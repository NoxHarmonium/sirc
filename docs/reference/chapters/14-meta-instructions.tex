\chapter{Meta-Instructions}
\label{ch:meta-instructions}

\section{Overview}

Meta-instructions (also called pseudo-instructions) are assembly mnemonics that don't directly correspond to unique
opcodes. Instead, they assemble to specific forms of existing instructions, providing more intuitive syntax for common
operations.

\section{NOOP -- No Operation}

\textbf{Syntax:} \texttt{NOOP}

\textbf{Assembles to:} \texttt{ADDI r1, \#0}

\textbf{Description:} Does nothing for 6 cycles. Useful for timing delays, instruction padding, or placeholders during development.

\textbf{Example:}
\begin{lstlisting}
NOOP                          ; Wait 6 cycles
NOOP
NOOP                          ; Total 18 cycles delay
\end{lstlisting}

\section{RETS -- Return from Subroutine}

\textbf{Syntax:} \texttt{RETS}

\textbf{Assembles to:} \texttt{LDEA p, (\#0, l)}

\textbf{Description:} Returns from a subroutine by copying the link register to the program counter. Used after \mnemonic{LJSR} or \mnemonic{BRSR}.

\textbf{Example:}
\begin{lstlisting}
my_function:
    ; ... function body ...
    RETS                      ; Return to caller
\end{lstlisting}

\section{WAIT -- Wait for Interrupt}

\textbf{Syntax:} \texttt{WAIT}

\textbf{Assembles to:} \texttt{COP \#0x1F00}

\textbf{Description:} Puts the CPU into a low-power wait state until an interrupt occurs. Sets the Waiting for Interrupt bit in the status register.

\textbf{Example:}
\begin{lstlisting}
main_loop:
    WAIT                      ; Sleep until interrupt
    ; ... process interrupt ...
    BRAN (#main_loop, p)
\end{lstlisting}

\section{RETE -- Return from Exception}

\textbf{Syntax:} \texttt{RETE}

\textbf{Assembles to:} \texttt{COP \#0x1A00}

\textbf{Description:} Returns from an exception handler by restoring the saved program counter and status register. Only valid in supervisor mode.

\textbf{Example:}
\begin{lstlisting}
exception_handler:
    ; ... handle exception ...
    RETE                      ; Return to interrupted code
\end{lstlisting}

\textit{(Note: Check assembler documentation for supported pseudo-instructions)}
