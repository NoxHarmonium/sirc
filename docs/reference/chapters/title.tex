\begin{titlepage}
    \centering

    \vspace*{2cm}

    {\Huge\bfseries\color{sircblue} SIRCULAR}

    \vspace{0.5cm}

    {\Large SIR Compute Using Load/store ALU and Registers}

    \vspace{2cm}

    {\LARGE\bfseries CPU Reference Manual}

    \vspace{0.5cm}

    {\large Featuring the SIRCIS Instruction Set}

    \vspace{3cm}

    \begin{tcolorbox}[colback=sirclightgray,colframe=sircblue,width=0.8\textwidth,arc=2mm]
        \centering
        \textbf{Technical Specifications}

        \vspace{0.3cm}

        \begin{tabular}{ll}
            \textbf{Architecture:} & 16-bit RISC Load/Store \\
            \textbf{Address Bus:} & 24-bit \\
            \textbf{Data Bus:} & 16-bit \\
            \textbf{Instruction Size:} & 32-bit (fixed) \\
            \textbf{Registers:} & 16 general/address registers \\
            \textbf{Execution:} & 6-stage pipeline \\
        \end{tabular}
    \end{tcolorbox}

    \vfill

    {\large\textit{Version 1.0}}

    \vspace{0.5cm}

    {\large\today}

\end{titlepage}

\clearpage
\thispagestyle{empty}
\vspace*{5cm}

\section*{About This Manual}

This reference manual provides comprehensive documentation for the SIRCULAR CPU architecture and the SIRCIS (SIRC Instruction Set) instruction set. It is intended for:

\begin{itemize}
    \item Assembly language programmers
    \item Compiler and toolchain developers
    \item Hardware designers and implementers
    \item System software developers
\end{itemize}

\section*{Document Conventions}

Throughout this manual, the following conventions are used:

\begin{itemize}
    \item Register names appear in typewriter font: \reg{r1}, \reg{ph}, \reg{sr}
    \item Hexadecimal values are prefixed with \texttt{0x}: \opcode{1A}
    \item Immediate values are prefixed with \texttt{\#}: \imm{123}
    \item Instruction mnemonics appear in bold typewriter font: \mnemonic{ADDI}
    \item Addressing modes use parentheses for indirection: \indirectimm{4}{a}
\end{itemize}

\section*{Acknowledgments}

The SIRCULAR CPU draws inspiration from several classic processor architectures of the late 1980s and early 1990s, including the Motorola 68000 series, MOS 6502, ARM6, and MIPS R2000/R3000.

\clearpage
