\chapter{Memory Access Instructions}
\label{ch:memory-instructions}

\section{Overview}

Memory access instructions transfer data between registers and memory. Following the load/store architecture, these are the only instructions that can access memory—all arithmetic and logical operations must work on registers.

The SIRCULAR provides two primary memory operations:
\begin{itemize}
    \item \mnemonic{LOAD} -- Read from memory into a register
    \item \mnemonic{STOR} -- Write from a register to memory
\end{itemize}

Both instructions support multiple addressing modes:
\begin{itemize}
    \item Indirect with immediate offset: \texttt{(\#offset, addr)}
    \item Indirect with register offset: \texttt{(reg, addr)}
    \item Post-increment: \texttt{(\#offset, addr)+}
    \item Pre-decrement: \texttt{-(\#offset, addr)}
\end{itemize}

\section{Effective Address Calculation}

For all memory operations, the effective address (EA) is calculated as:

\begin{equation}
\text{EA} = \text{AddressRegister} + \text{Offset}
\end{equation}

Where:
\begin{itemize}
    \item Address Register is one of: \reg{l}, \reg{a}, \reg{s}, or \reg{p}
    \item Offset is either an immediate value or register contents
\end{itemize}

The resulting 24-bit address is used to access memory.

\section{LOAD Instructions}

\begin{instructionbox}{LOAD -- Load from Memory}
\textbf{Opcodes:} 0x14 (Indirect Imm), 0x15 (Indirect Reg), 0x16 (Post-Inc Imm), 0x17 (Post-Inc Reg)

\textbf{Syntax:}
\begin{lstlisting}
LOAD rD, (#offset, addr)      ; rD = memory[addr + offset]
LOAD rD, (rS, addr)           ; rD = memory[addr + rS]
LOAD rD, (#offset, addr)+     ; rD = memory[addr + offset]; addr += offset
LOAD rD, (rS, addr)+          ; rD = memory[addr + rS]; addr += rS
\end{lstlisting}

\textbf{Description:}

Loads a 16-bit value from memory into the destination register. Post-increment forms update the address register after the load, useful for array traversal.

\textbf{Examples:}
\begin{lstlisting}
; Load from fixed offset
LOAD r1, (#4, a)              ; r1 = memory[a + 4]

; Load using variable offset
LOAD r2, (r3, a)              ; r2 = memory[a + r3]

; Array iteration
LOAD r4, (#0, a)+             ; r4 = *a; a += 2

; Stack pop
LOAD r5, (#0, s)+             ; r5 = *s; s += 2
\end{lstlisting}

\end{instructionbox}

\section{STOR Instructions}

\begin{instructionbox}{STOR -- Store to Memory}
\textbf{Opcodes:} 0x10 (Indirect Imm), 0x11 (Indirect Reg), 0x12 (Pre-Dec Imm), 0x13 (Pre-Dec Reg)

\textbf{Syntax:}
\begin{lstlisting}
STOR rS, (#offset, addr)      ; memory[addr + offset] = rS
STOR rS, (rD, addr)           ; memory[addr + rD] = rS
STOR rS, -(#offset, addr)     ; addr -= offset; memory[addr + offset] = rS
STOR rS, -(rD, addr)          ; addr -= rD; memory[addr + rD] = rS
\end{lstlisting}

\textbf{Description:}

Stores a 16-bit value from a register into memory. Pre-decrement forms update the address register before the store, useful for stack push operations.

\textbf{Examples:}
\begin{lstlisting}
; Store to fixed offset
STOR r1, (#8, a)              ; memory[a + 8] = r1

; Store using variable offset
STOR r2, (r3, a)              ; memory[a + r3] = r2

; Stack push
STOR r4, -(#2, s)             ; s -= 2; memory[s] = r4
\end{lstlisting}

\end{instructionbox}

\section{Common Memory Access Patterns}

\subsection{Structure Field Access}
\begin{lstlisting}
; struct { int16_t x, y, z; } point;
; 'a' points to point
LOAD r1, (#0, a)              ; r1 = point.x
LOAD r2, (#2, a)              ; r2 = point.y
LOAD r3, (#4, a)              ; r3 = point.z
\end{lstlisting}

\subsection{Array Iteration}
\begin{lstlisting}
; Process array of 16-bit values
; 'a' = array base, r7 = count
loop:
    LOAD r1, (#0, a)+         ; Load element, advance
    ; ... process r1 ...
    SUBI r7, #1
    BRAN|!= loop
\end{lstlisting}

\subsection{Stack Operations}
\begin{lstlisting}
; Push registers
STOR r1, -(#2, s)
STOR r2, -(#2, s)
STOR r3, -(#2, s)

; Pop registers (reverse order)
LOAD r3, (#0, s)+
LOAD r2, (#0, s)+
LOAD r1, (#0, s)+
\end{lstlisting}

\textit{(Additional memory instruction details would follow in a complete manual)}
