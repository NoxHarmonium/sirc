\chapter{Memory Access Instructions}
\label{ch:memory-instructions}

\section{Overview}

Memory access instructions transfer data between registers and memory. Following the load/store architecture, these are
the only instructions that can access memory—all arithmetic and logical operations must work on registers.

The SIRC-1 provides two primary memory operations:
\begin{itemize}
	\item \mnemonic{LOAD} -- Read from memory into a register
	\item \mnemonic{STOR} -- Write from a register to memory
\end{itemize}

Both instructions support multiple addressing modes:
\begin{itemize}
	\item Indirect with immediate offset: \texttt{(\#offset, addr)}
	\item Indirect with register offset: \texttt{(reg, addr)}
	\item Post-increment: \texttt{(\#offset, addr)+}
	\item Pre-decrement: \texttt{-(\#offset, addr)}
\end{itemize}

\textbf{Important:} When using indirect addressing modes (with memory access), \mnemonic{LOAD} and \mnemonic{STOR} instructions support optional shift operations on the data being loaded or stored. See Section~\ref{sec:memory-shifts} for details. Note that shifts are \textbf{not} supported when loading immediate values or copying register values directly.

\section{Effective Address Calculation}

For all memory operations, the effective address (EA) is calculated as:

\begin{equation}
	\text{EA} = \text{AddressRegister} + \text{Offset}
\end{equation}

Where:
\begin{itemize}
	\item Address Register is one of: \reg{l}, \reg{a}, \reg{s}, or \reg{p}
	\item Offset is either an immediate value or register contents
\end{itemize}

The resulting 24-bit address is used to access memory.

\section{LOAD Instructions}

\begin{instructionbox}{LOAD -- Load from Memory}
	\textbf{Opcodes:} 0x14 (Indirect Imm), 0x15 (Indirect Reg), 0x16 (Post-Inc Imm), 0x17 (Post-Inc Reg)

	\textbf{Syntax:}
	\begin{lstlisting}
LOAD rD, (#offset, addr)      ; rD = memory[addr + offset]
LOAD rD, (rS, addr)           ; rD = memory[addr + rS]
LOAD rD, (#offset, addr)+     ; rD = memory[addr + offset]; addr += 1
LOAD rD, (rS, addr)+          ; rD = memory[addr + rS]; addr += 1

; With optional shift (memory addressing modes only)
LOAD rD, (#offset, addr), shift      ; Load and shift
LOAD rD, (rS, addr), shift           ; Load and shift
LOAD rD, (#offset, addr)+, shift     ; Load, shift, then increment
LOAD rD, (rS, addr)+, shift          ; Load, shift, then increment
\end{lstlisting}

	\textbf{Description:}

	Loads a 16-bit value from memory into the destination register. Post-increment forms update the address register after
	the load, useful for array traversal.

	\textbf{Examples:}
	\begin{lstlisting}
; Load from fixed offset
LOAD r1, (#4, a)              ; r1 = memory[a + 4]

; Load using variable offset
LOAD r2, (r3, a)              ; r2 = memory[a + r3]

; Array iteration
LOAD r4, (#0, a)+             ; r4 = *a; a += 1

; Stack pop
LOAD r5, (#0, s)+             ; r5 = *s; s += 1

; Load and shift (for bit manipulation)
LOAD r3, (r2, a), LSL #2      ; r3 = memory[a + r2] << 2
LOAD r6, (#0, a), ASR #2      ; r6 = memory[a] >> 2 (signed)
\end{lstlisting}

\end{instructionbox}

\section{STOR Instructions}

\begin{instructionbox}{STOR -- Store to Memory}
	\textbf{Opcodes:} 0x10 (Indirect Imm), 0x11 (Indirect Reg), 0x12 (Pre-Dec Imm), 0x13 (Pre-Dec Reg)

	\textbf{Syntax:}
	\begin{lstlisting}
STOR (#offset, addr), rS      ; memory[addr + offset] = rS
STOR (rD, addr), rS           ; memory[addr + rD] = rS
STOR -(#offset, addr), rS     ; addr -= 1; memory[addr + offset] = rS
STOR -(rD, addr), rS          ; addr -= 1; memory[addr + rD] = rS

; With optional shift (memory addressing modes only)
STOR (#offset, addr), rS, shift      ; Shift and store
STOR (rD, addr), rS, shift           ; Shift and store
STOR -(#offset, addr), rS, shift     ; Decrement, shift, and store
STOR -(rD, addr), rS, shift          ; Decrement, shift, and store
\end{lstlisting}

	\textbf{Description:}

	Stores a 16-bit value from a register into memory. Pre-decrement forms update the address register before the store,
	useful for stack push operations.

	\textbf{Examples:}
	\begin{lstlisting}
; Store to fixed offset
STOR (#8, a), r1              ; memory[a + 8] = r1

; Store using variable offset
STOR (r3, a), r2              ; memory[a + r3] = r2

; Store with shift (data shifted before storing)
STOR (r2, a), r1, LSL #1      ; memory[a + r2] = r1 << 1
STOR (#0, a), r1, LSR #3      ; memory[a] = r1 >> 3

; Stack push
STOR -(#0, s), r4             ; s -= 1; memory[s] = r4
\end{lstlisting}

\end{instructionbox}

\section{Common Memory Access Patterns}

\subsection{Structure Field Access}
\begin{lstlisting}
; struct { int16_t x, y, z; } point;
; 'a' points to point
LOAD r1, (#0, a)              ; r1 = point.x
LOAD r2, (#1, a)              ; r2 = point.y
LOAD r3, (#2, a)              ; r3 = point.z
\end{lstlisting}

\subsection{Array Iteration}
\begin{lstlisting}
; Process array of 16-bit values
; 'a' = array base, r7 = count
loop:
    LOAD r1, (#0, a)+         ; Load element, advance
    ; ... process r1 ...
    SUBI r7, #1
    BRAN|!= loop
\end{lstlisting}

\subsection{Stack Operations}
\begin{lstlisting}
; Push registers
STOR -(#0, s), r1
STOR -(#0, s), r2
STOR -(#0, s), r3

; Pop registers (reverse order)
LOAD r3, (#0, s)+
LOAD r2, (#0, s)+
LOAD r1, (#0, s)+
\end{lstlisting}

\section{Shift Operations with Memory Instructions}
\label{sec:memory-shifts}

When using indirect addressing modes (memory access), both \mnemonic{LOAD} and \mnemonic{STOR} instructions support
optional shift operations. This feature allows data to be shifted during the load or store operation, which is
particularly useful for bit manipulation and working with the CPU's word-addressed memory architecture.

\subsection{LOAD with Shift}

When loading from memory, the data read from memory is shifted \textbf{after} being loaded but \textbf{before} being
written to the destination register:

\begin{lstlisting}
LOAD r3, (r2, a), LSL #2      ; r3 = (memory[a + r2]) << 2
LOAD r4, (#0, a), LSR #4      ; r4 = (memory[a]) >> 4
LOAD r5, (r2, a), ASL #3      ; r5 = (memory[a + r2]) << 3
LOAD r6, (#0, a), ASR #2      ; r6 = (memory[a]) >> 2 (sign-extend)
\end{lstlisting}

\subsection{STOR with Shift}

When storing to memory, the source register value is shifted \textbf{before} being written to memory. The original
register value is \textbf{not} modified:

\begin{lstlisting}
STOR (r2, a), r1, LSL #1      ; memory[a + r2] = r1 << 1 (r1 unchanged)
STOR (#0, a), r1, LSR #3      ; memory[a] = r1 >> 3 (r1 unchanged)
STOR (r2, a), r1, ASL #2      ; memory[a + r2] = r1 << 2 (r1 unchanged)
STOR (#0, a), r1, ASR #4      ; memory[a] = r1 >> 4 (r1 unchanged)
\end{lstlisting}

\subsection{Supported Shift Types}

All standard shift types are supported:
\begin{itemize}
	\item \textbf{LSL} -- Logical Shift Left (fill with zeros)
	\item \textbf{LSR} -- Logical Shift Right (fill with zeros)
	\item \textbf{ASL} -- Arithmetic Shift Left (same as LSL)
	\item \textbf{ASR} -- Arithmetic Shift Right (sign-extend)
	\item \textbf{RTL} -- Rotate Left (through carry)
	\item \textbf{RTR} -- Rotate Right (through carry)
\end{itemize}

\subsection{Shift Limitations}

\textbf{Important:} Shift operations are \textbf{only} supported when using indirect addressing modes (accessing memory). The following \mnemonic{LOAD} modes do \textbf{not} support shifts:
\begin{itemize}
	\item Loading immediate values: \texttt{LOAD r1, \#0x1234} -- No shift allowed
	\item Copying registers: \texttt{LOAD r1, r2} -- No shift allowed
\end{itemize}

For these non-memory operations, use regular ALU instructions with shifts instead (e.g., \texttt{ADDI r1, \#0, LSL
	\#2}).

\subsection{Use Cases}

\subsubsection{Word-Addressed Memory}

The SIRC-1 uses word addressing, so shifting during load/store can save cycles when converting between byte and word
addresses:

\begin{lstlisting}
; Convert byte offset to word address and load
LOAD r3, (r2, a), LSR #1      ; Divide byte offset by 2
\end{lstlisting}

\subsubsection{Bit Manipulation}

Shifting during memory access enables efficient bit manipulation:

\begin{lstlisting}
; Extract upper byte from memory
LOAD r1, (#0, a), LSR #8      ; r1 = (memory[a] >> 8) & 0xFF

; Store scaled value
STOR (#0, a), r2, LSL #2      ; memory[a] = r2 * 4
\end{lstlisting}

\subsubsection{Combined with Auto-Increment/Decrement}

Shifts work seamlessly with post-increment and pre-decrement modes:

\begin{lstlisting}
; Load, shift, and advance pointer
LOAD r3, (r2, a)+, LSL #1     ; r3 = (memory[a + r2]) << 1; a += 1

; Decrement, shift, and store
STOR -(r2, a), r1, LSR #2     ; a -= 1; memory[a + r2] = r1 >> 2
\end{lstlisting}

\subsection{Status Flags}

\textbf{Note:} Unlike shifts in ALU instructions, shifts performed during \mnemonic{LOAD} and \mnemonic{STOR} operations can \textbf{not} update the status register flags. If you need flag updates based on a shift result, use an ALU instruction or the \mnemonic{SHFT} meta-instruction (see Chapter~\ref{ch:meta-instructions}).
