\chapter{Introduction}
\label{ch:introduction}

\section{Overview}

The SIRC-1 is a 16-bit RISC processor ideal for general purpose computing.

\noindent It implements the SIRCIS (SIRC Instruction Set) instruction set architecture, which provides a comprehensive set of operations while maintaining a clean, consistent design philosophy.

\section{Design Philosophy}

\subsection{RISC Principles}

The SIRC-1 adheres to Reduced Instruction Set Computer (RISC) design principles:

\begin{itemize}
	\item \textbf{Load/Store Architecture:} All ALU operations work exclusively on registers. Memory access is performed only through dedicated load and store instructions.

	\item \textbf{Fixed Instruction Length:} All instructions are exactly 32 bits (2 words) in length, simplifying instruction fetch and decode logic.

	\item \textbf{Simple Instruction Formats:} Only four instruction formats are supported: Implied, Immediate, Short Immediate with Shift, and Register.

	\item \textbf{Register-Rich:} Sixteen addressable registers provide ample workspace for computations without frequent memory access.

	\item \textbf{Consistent Execution Time:} All instructions execute in exactly 6 clock cycles, eliminating the need for microcode and simplifying hardware implementation.
\end{itemize}

\subsection{Design Goals}

The SIRC-1 was designed with several key goals in order of importance:

\begin{description}
	\item[Simplicity] The CPU should be simple enough to be implemented in hardware without requiring
	      complex microcode engines or multi-level instruction decoders.

	\item[Developer Ergonomics] It should have powerful instruction encodings that include conditional execution and bit shifting
    that can help reduce the number of instructions and branches to keep track of.

	\item[Performance] While simple, the design should allow for reasonable performance through efficient instruction encoding
	      and the potential for future enhancements like pipelining.

	\item[Memory Protection] Basic privilege levels and memory segmentation provide rudimentary protection for system software.
\end{description}

\subsection{Applications}

The SIRC-1 is designed for ideal for general purpose computing, such as the core of a microcomputer.

\noindent It should not be used in memory constrained environments, as the fixed length instructions and lack of byte addressing can potentially mean larger program sizes.


\section{Key Features}

\subsection{Architecture Characteristics}

\begin{table}[H]
	\centering
	\begin{tabular}{ll}
		\toprule
		\textbf{Feature}            & \textbf{Specification}      \\
		\midrule
		Data Width                  & 16-bit                      \\
		Address Bus                 & 24-bit (16MB address space) \\
		Instruction Size            & 32-bit fixed                \\
		General Purpose Registers   & 7 (r1--r7)                  \\
		Address Register Pairs      & 4 (l, a, s, p)              \\
		Total Addressable Registers & 16                          \\
		Instruction Formats         & 4                           \\
		Addressing Modes            & 8                           \\
		Condition Codes             & 16                          \\
		\bottomrule
	\end{tabular}
	\caption{SIRC-1 CPU Specifications}
	\label{tab:specs}
\end{table}

\subsection{Execution Model}

The SIRC-1 employs a six-stage execution sequence (one stage per clock):

\begin{enumerate}
	\item \textbf{Instruction Fetch (First Word)} -- Fetch bits 31--16 from memory
	\item \textbf{Instruction Fetch (Second Word)} -- Fetch bits 15--0 from memory
	\item \textbf{Decode and Register Fetch} -- Decode the instruction and read source registers
	\item \textbf{Execute and Address Calculation} -- Perform ALU operation or calculate effective address
	\item \textbf{Memory Access} -- Read from or write to memory (or NOP for register operations)
	\item \textbf{Write Back} -- Write results to destination register
\end{enumerate}

This fixed six-stage model means that all instructions take exactly 6 clock cycles to complete, regardless of their
complexity. While this may seem wasteful for simple register operations that don't require memory access, it
dramatically simplifies the hardware design by eliminating the need for variable-length execution cycles or microcode.

\subsection{Coprocessors}

The SIRC-1 is made up of multiple coprocessors (also known as modules) that all share the same address/data bus and register file.
Only one coprocessor can be active at one time and are usually activated using the COP instructions (except for the "exception unit" which can be activated when physical pins are asserted on the chip).
All implementations of the SIRC-1 will have the "processing unit" as coprocessor 0x0 and "exception unit" as coprocessor 0x1.
Other coprocessors are optional and will vary depending on which SIRC-1 model you have.
The instruction set supports up to 16 coprocessors.

\begin{table}[H]
	\centering
	\begin{tabular}{llll}
		\toprule
		\textbf{ID} & \textbf{Name}        & \textbf{Optional}        & \textbf{Purpose}                                           \\
		\midrule
		0x0           & Processing Unit    & No                       & Executes instructions                                      \\
		0x1           & Exception Unit     & No                       & Handles interrupts and exceptions                          \\
		0x2           & DMA Unit           & Yes                      & Transfers large amounts of data via the bus                \\
		0x3           & Maths Unit         & Yes                      & Performs mathematical operations in hardware               \\
		\bottomrule
	\end{tabular}
	\caption{List of coprocessors}
	\label{tab:coprocessor-list}
\end{table}

Each coprocessor can execute 16 opcodes, the first 8 (0x0-0x7) can be called in any mode. The last 8 (0x8-0xF) can only be called in supervisor mode.

\subsection{Privilege Levels}

The SIRC-1 supports two privilege levels:

\begin{description}
	\item[Supervisor Mode] Full access to all registers and instructions. Can set the high bytes of address register pairs,
	      access privileged status register bits, and execute privileged coprocessor operations.

	\item[Protected Mode] Restricted access for user programs. Cannot write to the high bytes of address register pairs
	      (providing memory segmentation/protection), cannot access privileged status register bits, and cannot execute
	      privileged coprocessor instructions.
\end{description}

The privilege level is controlled by the Protected Mode bit in the status register.

\section{Memory Organization}

\subsection{Address Space}

The SIRC-1 provides a 24-bit address space, allowing access to 16 megabytes of memory. Addresses are formed by
combining the 8-bit high byte and 16-bit low word of address register pairs:

\begin{equation}
	\text{Address} = (\text{RegHigh}[7:0] \ll 16) \mid \text{RegLow}[15:0]
\end{equation}

Note that the upper 8 bits of the high register (bits 15--8) are not used for addressing but are available for storage,
enabling techniques like tagged pointers.

\subsection{Memory Map}

The CPU expects certain memory regions to be defined:

\begin{itemize}
	\item \textbf{Reset Vector} (\addr{0x000000}): Initial program counter after reset
	\item \textbf{Fault Vectors} (\addr{0x000001}--\addr{0x000007}): Fault vectors
	\item \textbf{Hardware Vectors} (\addr{0x000010}--\addr{0x000050}): Hardware exception vectors (offset by 0x10 due to hardware implementation)
	\item \textbf{User Vectors} (\addr{0x000060}--\addr{0x0000FF}): User exception vectors
\end{itemize}

Beyond these reserved regions, the memory layout is flexible and can be defined by system software.

\subsection{Alignment}

\begin{itemize}
	\item All instructions must be aligned on 4-byte boundaries (word-aligned)
	\item Data can be accessed at any 2-byte boundary (halfword-aligned)
	\item Unaligned instruction fetches trigger an alignment fault exception
\end{itemize}

\section{Document Organization}

This manual is organized into three main parts:

\begin{description}
	\item[Part I: Architecture Overview] Covers the register model, status register, and overall CPU architecture.

	\item[Part II: Instruction Set Architecture] Details instruction formats, addressing modes, shift operations, and condition
	      codes.

	\item[Part III: SIRCIS Instruction Reference] Provides detailed documentation for each instruction, including encoding,
	      operation, and examples.
\end{description}

Appendices provide quick reference materials including complete opcode maps, timing diagrams, and notes on undocumented
instructions.
